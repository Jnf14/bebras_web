\documentclass[a4paper,12pt]{article}

\usepackage[utf8]{inputenc}
\usepackage[T1]{fontenc}
\usepackage[ngerman]{babel}
\usepackage{lmodern}
\usepackage[margin=2mm,paperwidth=12.1cm,paperheight=5.1cm]{geometry}
\usepackage{microtype}
\usepackage{fontawesome}

\usepackage{tikz}

\usetikzlibrary{arrows,positioning,automata}
\tikzset{
  every initial by arrow/.style={initial text={}},
  every state/.style={semithick},
  accepting by double/.style={double distance=.4mm},
  font=\small,
  >=stealth', %'
  auto
}

\renewcommand{\familydefault}{\sfdefault}

\pagestyle{empty}

\setlength{\parindent}{0em}
\setlength{\parskip}{1em}

\begin{document}

%  \begin{center}
%    \bfseries\Huge
%    Lieblings-Biber-Lieder
%  \end{center}
%
%  \bigskip
%
%  Biberin Lisa singt gerne Biber-Lieder -- aber nicht beliebige Biber-Lieder,
%  sondern nur Lisa-Lieblings-Biber-Lieder.  Alle ihre Lieblings-Biber-Lieder
%  gehen so:
%  \begin{quote}
%    \itshape\hspace{-1.6cm}\scalebox{3}{\faMusic}\hspace{0.33cm}
%    Dum-da-da Dum-da-da \quad Bang \quad Dum-da-da Dum-da-da
%  \end{quote}
%  oder auch 
%  \begin{quote}
%    \itshape\hspace{-1.6cm}\scalebox{3}{\faMusic}\hspace{0.33cm}
%    Dum-da-da Dum-da-da \quad Bang \quad Dum-da-da Dum-da-da \quad Bang \\
%    Dum-da-da Dum-da-da \quad Bang \quad Dum-da-da Dum-da-da
%  \end{quote}
%  Lisa-Lieblings-Biber-Lieder bestehen also aus der Wiederholung von "`Dum-da-da Dum-da-da"',
%  wobei immer ein "`Bang"' zwischen zwei Wiederholungen kommt.
%  Um sich das merken zu k\"onnen, erstellt Lisa folgendes Diagramm:
%  \begin{center} 
%    \begin{tikzpicture}[node distance=3cm]
%      \node[state] (0)              {Start};
%      \node[state] (1) [right of=0] {};
%      \node[state] (2) [right of=1] {End};
%      \path[->] (0) edge[bend left] node {Dum-da-da} (1)
%                (1) edge[bend left] node {Dum-da-da} (2)
%                (2) edge[bend left] node {Bang} (0);
%    \end{tikzpicture}
%  \end{center} 
%  Um zu singen, beginnt Lisa bei "`Start"' und folgt dann den Pfeilen, um
%  die Abfolge der Worte richtig zu singen.  Das Lied beenden darf sie nur, wenn
%  sie auf "`End"' kommt.
%
%  Nach einer Zeit wird ihr langweilig und sie m\"ochte statt der Wiederholung
%  von "`Dum-da-da"' auch noch "`Wupp-di-doo lalala"' singen k\"onnen.
%  Um beides zu kombinieren erstellt folgendes Diagramm: 
%  \begin{center} 
%    \begin{tikzpicture}[node distance=3cm]
%      \node[state] (0)              {Start};
%      \node[state] (1) [right of=0] {};
%      \node[state] (3) [left of=0]  {};
%      \node[state] (2) [right of=1] {End};
%      \path[->] (0) edge[bend left]  node {Dum-da-da} (1)
%                (1) edge[bend left]  node {Dum-da-da} (2)
%                (0) edge[bend right] node[swap] {Wupp-di-doo} (3)
%                (3) edge[bend right] node[swap] {lalala} (0)
%                (2) edge[bend left]  node {Bang} (0);
%    \end{tikzpicture}
%  \end{center} 
%
%  Wieder beginnt sie bei "`Start"' und endet bei "`End"'.
%  \begin{quote}
%    \itshape\hspace{-1.6cm}\scalebox{3}{\faMusic}\hspace{0.33cm}
%    Wupp-di-doo lalala \quad Wupp-di-doo lalala \\
%    Dum-da-da Dum-da-da \quad Bang \quad Dum-da-da
%  \end{quote}
%  oder auch
%  \begin{quote}
%    \itshape\hspace{-1.6cm}\scalebox{3}{\faMusic}\hspace{0.33cm}
%    Dum-da-da Dum-da-da \quad Bang \quad Wupp-di-doo lalala \quad Wupp-di-doo lalala \\
%    Dum-da-da Dum-da-da \quad Bang \quad Wupp-di-doo lalala \\
%    Dum-da-da Dum-da-da \quad Bang \quad Dum-da-da Dum-da-da 
%  \end{quote}
%
%  \section*{Aufgaben}
%
%  Hilf Lisa, neue Lieder zu singen, die mit Diagrammen dargestellt werden.
%
%  \subsection*{Aufgabe 1}
%
%  Betrachte das folgende Diagramm und die Lieder, die mit diesem gesungen
%  werden k\"onnen. 
%  
%  \begin{center}
%    \begin{tikzpicture}[node distance=3cm]
%      \node[state] (0)             {Start};
%      \node[state] (1) [left of=0] {};
%      \node[state] (2) [left of=1] {};
%      \node[state] (3) [left of=2] {End};
%      \path[->] (0) edge[bend right] node[swap] {Jup-pi} (1)
%                (1) edge[bend right] node[swap] {Dup-pi} (2)
%                (2) edge[bend right] node[swap] {Dup-pi} (1)
%                (2) edge[bend right] node[swap] {Duuu} (3)
%                (3) edge[bend right] node[swap] {Juhuu} (0);
%    \end{tikzpicture}
%  \end{center}
%
%  Welches der drei Diagramme ist f\"ur dieselben Lieder?
%
%  \begin{enumerate}
%    \item[]\ \\
%    \begin{tikzpicture}[node distance=4cm]
%      \node[state] (0)                  {Start};
%      \node[state] (1) [right of=0]      {};
%      \node[state] (2) [left=4cm of 0]  {};
%      \node[state] (2a) [above=2cm of 2]  {};
%      \node[state] (3) [below right=2cm and 1cm of 2] {End};
%      \path[->] (0) edge[bend right] node[swap] {Jup-pi} (1)
%                (1) edge[bend right] node[swap] {Dup-pi Dup-pi Dup-pi} (2)
%                (2) edge[bend right] node[swap] {Dup-pi} (2a)
%                (2a) edge[bend right] node[swap] {Dup-pi} (2)
%                (2) edge[bend right] node[swap] {Duuu} (3)
%                (0) edge[bend left] node {Jup-pi Dup-pi} (2)
%                (3) edge[bend right] node[swap] {Juhuu} (0);
%    \end{tikzpicture}
%
%    \newpage
%
%    \item[]\ \\
    \begin{tikzpicture}[node distance=4cm]
      \node[state] (0)                  {Start};
      \node[state] (1) [left of=0]      {};
      \node[state] (2) [left=4cm of 1]  {};
      \node[state] (3) [below=1cm of 2] {End};
      \path[->] (0) edge[bend right] node[swap] {Jup-pi} (1)
                (1) edge[bend right] node[swap] {Dup-pi Dup-pi Dup-pi} (2)
                (2) edge[bend right] node[swap] {Doo} (3)
                (0) edge[bend left] node {Jup-pi Dup-pi} (2)
                (3) edge[bend right] node[swap] {Yahoo} (0);
    \end{tikzpicture}
%
%    \item[]\ \\
%    \begin{tikzpicture}[node distance=5cm]
%      \node[state] (0)                  {Start};
%      \node[state] (1) [left of=0]      {};
%      \node[state] (2) [left=3cm of 1]  {};
%      \node[state] (3) [above=2cm of 2] {End};
%      \path[->] (0) edge[bend right] node[swap] {Jup-pi} (1)
%                (1) edge[bend right] node[swap] {Dup-pi Dup-pi} (2)
%                (2) edge[bend left] node[] {Duuu} (3)
%                (0) edge[bend left] node {Jup-pi Dup-pi} (2)
%                (3) edge[bend left] node[swap] {Juhuu} (0);
%    \end{tikzpicture}
%  \end{enumerate}
% 
%  \subsection*{Aufgabe 2}
%
%  Drei der folgenden Diagramme sind f\"ur das gleiche Lied.  Welches ist f\"ur ein
%  anderes Lied, d.\,h.\ welches Diagramm passt nicht zu den anderen?
%  
%  \begin{enumerate}
%    \item[]\ \\
%    \begin{tikzpicture}[node distance=3cm]
%      \node[state] (0)              {Start};
%      \node[state] (1) [right of=0] {};
%      \node[state] (2) [right of=1] {};
%      \node[state] (3) [right of=2] {};
%      \node[state] (4) [right of=3] {End};
%      \path[->] (0) edge[bend left] node {La} (1)
%                (1) edge[bend left] node {La} (2)
%                (2) edge[bend left] node {La} (3)
%                (3) edge[bend left] node {Wupp-di-dooo} (1)
%                (3) edge[bend left] node {Schwang} (4);
%    \end{tikzpicture}
%    \item[]\ \\
%    \begin{tikzpicture}[node distance=3cm]
%      \node[state] (0)              {Start};
%      \node[state] (1) [left of=0] {};
%      \node[state] (3) [left of=1] {};
%      \node[state] (4) [left of=3] {End};
%      \path[->] (0) edge[bend right] node[swap] {La} (1)
%                (1) edge[bend right] node[swap] {La La} (3)
%                (3) edge[bend right] node[swap] {Wupp-di-dooo} (1)
%                (3) edge[bend right] node[swap] {Schwang} (4);
%    \end{tikzpicture}
%    \item[]\ \\
%    \begin{tikzpicture}[node distance=3cm]
%      \node[state] (0)              {Start};
%      \node[state] (1) [left of=0] {};
%      \node[state] (2) [left of=1] {};
%      \node[state] (4) [left of=2] {End};
%      \path[->] (0) edge[bend right] node[swap] {La La} (1)
%                (1) edge[bend right] node[swap] {La} (2)
%                (2) edge[bend right] node[swap] {Wupp-di-dooo} (0)
%                (2) edge[bend right] node[swap] {Schwang} (4);
%    \end{tikzpicture}
%  \end{enumerate}
%
%  \subsection*{Aufgabe 3}
%
%  Betrachte das folgende Diagramm. 
%  
%  \begin{center}
%    \begin{tikzpicture}[node distance=3cm]
%      \node[state] (0)                  {Start};
%      \node[state] (1) [right of=0]     {};
%      \node[state] (2) [left of=0]      {};
%      \node[state] (3) [below of=1]     {};
%      \node[state] (4) [below of=2]     {};
%      \node[state] (5) [below=5cm of 0] {End};
%      \path[->] (0) edge[bend left] node {Lala} (1)
%                (0) edge[bend right] node[swap] {Dum-di} (2)
%                (1) edge[bend left] node[] {Dada} (3)
%                (2) edge[bend right] node[swap] {Dum} (4)
%                (4) edge[bend right] node[swap] {Dada} (2)
%                (4) edge[bend right] node[swap] {Lalala} (1)
%                (3) edge[bend left] node[] {Diddel-di} (4)
%                (4) edge[bend right] node[swap] {Dada-dam} (5);
%    \end{tikzpicture}
%  \end{center}
%
%  Welches der folgenden Lieder kann mit dem Diagramm nicht gesungen werden? 
%
%  \hspace{-7mm}\begin{minipage}{1.6cm}
%    \scalebox{3}{\faMusic}
%  \end{minipage}
%  \begin{minipage}{14cm}
%    \begin{enumerate}
%      \item \textit{Lala Dada Diddel-di Dada Dum Dada Dum Dada-dam}
%      \item \textit{Dum-di Dum Dada Dada-dam}
%      \item \textit{Dum-di Dum Dada Dum Lalala Dada  Diddel-di Dada Dum Dada-dam}
%    \end{enumerate}
%  \end{minipage}
%
%  \section*{L\"osungen zu den Aufgaben}
%
%  \textbf{Aufgabe 1.} Das erste Diagramm ist f\"ur dieselben Lieder.  Der wesentliche
%    Punkt ist, dass mit dem Mittelteil das "`Dup-pi"' jeweils ein Mal oder drei Mal wiederholt
%    werden kann.
%
%  \smallskip
%
%  \textbf{Aufgabe 2.} Das erste und das dritte Diagramm sind f\"ur dieselben Lieder.
%    Hier ist der wesentliche Punkt, dass auf ein "`La La"' immer ein "`Wupp-di-dooo"' oder
%    am End ein "`Schwang"' folgen muss.
%  
%  \smallskip
%
%  \textbf{Ausgabe 3.}  Das zweite Lied kann nicht gesungen werden, da auf das "`Dada"'
%    kein "`Dada-dam"' folgen kann. 
%
%  
%  \enlargethispage{1cm}
%
%  \section*{Das ist Informatik}
%
%  In der Informatik geht es zu einem wesentlichen Teil um das Finden von Strukturen.
%  In den Aufgaben betrachten wir strukturierte Texte (Lieder), die nach einem festen
%  Regelsystem aufgebaut sind.  Die erzeugenden Mechanismen ("`Diagramme"') sind hier sogenannte
%  \emph{Endliche Automaten}, die eine andere Darstellungsform f\"ur \emph{regul\"are Ausdr\"ucke} sind.
%
%  Diese spielen eine sehr wichtige Rolle in der Textverarbeitung und der
%  Formalisierung von Programmiersprachen, also Sprachen, die der Computer "`verstehen"'
%  kann. 

\end{document}
